\chapter{Approach}\label{chapter:evaluation}
The goal of this thesis is to integrate a file backed shared memory system into the existing sys-sage functionality,
allowing the transfer of topologies across processes and nodes within the HPC cluster.

This can be achieved, by creating memory mapped files using the \lstinline{mmap()} syscall and copying the topology or component subtree into the shared memory region.

As described in \autoref{section:related_work}, there are different approaches to implementing shared memory regions using \lstinline{mmap()}. For the purposes of this thesis,
an approach similar to the \emph{sharedstructures} implementation was chosen, meaning the implementation uses offset based pointers rather than enforcing identical virtual memory addresses across all processes.

This design was chosen, as it cannot necessarily be guaranteed that a sufficiently large memory section can be found for the shared topology, which could negatively affect the reliability of the system.
Additionally, at the time of exporting the topology into the shared memory region, it is not always known, which processes will need to import the topology.

However, using offset based pointers also means that the topology will have to be deconstructed before being copied into the shared memory region, as some internal components, such as vectors or maps,
rely on regular pointers that will not work within the shared region.

Due to the topology being deconstructed before being copied to the shared region,
it cannot be used directly in the shared memory block and needs to be reconstructed in the local memory of the importing process before being usable again.

This unfortunately makes it impossible for multiple processes to share a topology, each process being able to update the component tree as needed.
Instead, processes can only send snapshots of a topology to another process over the shared memory region. After the topology has been recreated in the importing processes local memory,
each process can make updates to their own version of the topology, however, updates are not shared between processes. If a shared topology needs to be updated across all processes,
the updated topology needs to be shared entirely anew.


\section{Capabilities}
In essence, the shared memory implementation of this thesis consists of two main functions.

With \lstinline|export_topology()| the exporting process can create a new shared memory region in a provided file location and
copy a chosen topology or component subtree including DataPaths and attribs into the newly created file backed memory region.
The \lstinline|export_topology()| function also allows the user to choose, which of the topologies attribs will be exported into the shared memory region.

Any process wishing to use the shared topology can then use \lstinline|import_topology()| to recreate the topology in its local memory.
