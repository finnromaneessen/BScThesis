%TODO Use lstinline instead of emph for code?

\chapter{Implementation}\label{chapter:Implementation}

The part of the sys-sage library implemented in this thesis enables users to share component subtrees or whole topologies between processes of a compute node by using shared memory regions.

To achieve this, all components of the given subtree, including its attributes and all DataPaths, are copied into a memory region shared between the involved processes.
The component tree and DataPath graph are then recreated in the memory of the receiving process.

\section{Capabilities and Usage}
If at any point during the lifetime of a sys-sage topology,

%\section{Limitations}
\section{Shared Memory}
The data sharing aspect of the implementation is realized using shared memory regions. These regions are created by opening files and mapping them using \lstinline{mmap()}.

\lstinline{mmap()} is a syscall that creates file backed memory mappings in the program's virtual address space that can be opened by multiple processes at once.
The mapped file can then be used just like any regular memory location. \cite{crotty22-mmap}

Using the \emph{MAP\_FIXED} flag when creating a \lstinline{mmap()} backed memory location will guarantee the virtual memory addresses to be equal across processes.
However, if the necessary memory location is not available in the current process, the mapping will fail, which could potentially have a major impact on the reliability of the library,
depending on the total available memory and its current utilization. [Quote manpage mmap] %TODO

Consequently, the \emph{MAP\_FIXED} flag is not used for the purposes of this thesis to achieve higher reliability when sharing component trees between processes.
As a result, the virtual memory addresses of the shared regions are not identical across processes.

Due to this, sharing pointers to addresses in the shared memory region between processes will not work, as the as the referenced location will have a different address in another process.
Instead, offset based pointers have to be used to reference shared memory locations.

In the sys-sage shared memory implementation, all offsets for pointers are calculated relative to the top of the shared region.
While this might not be possible for more general uses of offset pointers, as the start of the memory location might not always be known,
it is practical for the particular use-case of this thesis, since  memory regions are always handled as a whole and importing only parts of a shared component tree is not supported.

Calculating the offsets based on a shared, fixed location has the advantage that the offset pointers can be used more similarly to regular pointers and don't need to be recalculated when shared.
This means the location of components or DataPaths within the shared memory region can be compared or referenced without ambiguity or confusion about the base of the offset.

The lifetime of the shared memory region is handled by a \emph{SharedMemory} object, as shown in \autoref{lst:sharedmemory}.

%TODO Wrap in Figure?
\begin{lstlisting}[language=c++, numbers=left, caption=SharedMemory Class, captionpos=b, label={lst:sharedmemory}]
    class SharedMemory {
      public:
        void* mem;
        char* cur;
        size_t size;
        ...

        SharedMemory(std::string path, size_t size);
        SharedMemory(std::string path);
        ~SharedMemory() { munmap(mem, size); }

      private:
        std::string path;
    };
\end{lstlisting}

Apart from the \emph{path} and \emph{size} size variables, which are used mainly in the creation and destruction of the shared memory region,
the \emph{SharedMemory} class consists of two pointers, \emph{mem} and \emph{cur}.
The \emph{mem} pointer always points to the top of the memory region, whereas the \emph{cur} pointer marks the current location to write or read from while importing or exporting a topology.

When a shared memory region is first created by the process sharing the topology, the path to the file used in the mapping as well as the total size needed have to be known.
The allocated size is then written to the start of the mapped file, to be read by other processes when importing the topology.
The file-backed memory region can then be used to export the topology until the \emph{SharedMemory} destructor is called and the file is unmapped using \lstinline{munmap()}.

The process importing the topology then uses the constructor in line 9 of \autoref{lst:sharedmemory}, which opens and maps the previously created file,
reads the total size of the data as written by the first process and then remaps the file to that size.
This has the advantage that the total size of the shared topology is always known to the importing process, without having to be provided separately.
To share a topology, only the path to the memory mapped file needs to be provided, all other information can be read from the file, which simplifies the API and makes it easier
for users to share topologies without much inter-process communication needed.


[Size Calculation]%TODO

\section{Components}
Topologies consist mainly of \emph{components}, which are organized as a hierarchical tree structure.
Each component represents a certain part of the hardware such as a CPU or cache.
Depending on the type of hardware the component represents, there are different subclasses of Components that can store specific hardware information such as the size of a cache.
Although there is no formal requirement, the top of the component tree is usually represented by a component of class \emph{Topology}, a subclass of \emph{Component},
which stores no additional values.
\autoref{lst:component_class} shows part of the sys-sage component class implementation.

\begin{lstlisting}[language=c++, numbers=left, caption=Component Class, captionpos=b, label={lst:component_class}]
    class Component {
      public:
        map<string, void*> attrib;

      protected:
        int id;
        string name;

        const int componentType;
        vector<Component*> children;
        Component* parent { nullptr };
        vector<DataPath*> dp_incoming;
        vector<DataPath*> dp_outgoing;
    };
\end{lstlisting}

As shown in \autoref{lst:component_class}, the component tree structure is created by the vector of component pointers in line 10 for the children, as well as a component pointer
for the parent.
Apart from that, the \emph{componentType} variable indicates, which component subclass and therefore which type of hardware is represented by the component.
\emph{Id} and \emph{name} store additional information to identify the component and underlying hardware.

The \emph{attrib map} enables users to attach arbitrary data to a component by associating it with a key string. This allows for a high degree of customization,
as the user can attach any data and update it dynamically as needed.

The vectors \emph{dp\_incoming} and \emph{dp\_outgoing} in lines 12 and 13 are used to store pointers to DataPaths associated with the component.

\subsection{Exporting Components}\label{subsection:export_components}
To export the topology into the shared memory region, the component tree structure including all \emph{attribs} and DataPaths needs to be transformed into one contiguous memory block.

Copying the component tree into the shared memory region is performed as follows:
\begin{enumerate}
    \item Determining the size of the component based on its \emph{componentType}.
    \item Copying the \emph{Component} object into the shared region using \lstinline{memcpy()} and the size determined previously.
    \item Copying the \emph{attrib} map.
    \item Copying the \emph{children} vector.
    \item Recursively copying the children.
\end{enumerate}

\begin{figure}[!ht] %TODO Fix position
    \includegraphics[scale=0.15]{images/component_memory.jpg} %TODO Replace graphic
    \centering
    \caption{Component Tree in Shared Memory}
    \label{figure:component_memory}
\end{figure}

\autoref{figure:component_memory} illustrates how the component tree is transformed into a single contiguous memory segment.
The \emph{Component} object is copied first, followed by its \emph{attribs} and the data segment of the \emph{children} vector.
It is important to note that the component pointers of the \emph{children} vector need to be replaced with the respective offsets of the children in the shared memory region,
as the virtual memory addresses will be different in each process.

To achieve this, the children are recursively exported and their offsets written to the \emph{children} vectors data array.

\subsection{Copying Vectors}
Since vectors use a pointer to the contiguous heap memory location storing the underlying data, simply copying the vector object into the shared memory region will not work,
as the virtual memory address of the data will be different in other processes. [FIND QUOTE] %TODO

To circumvent this issue, the components \emph{children} vector needs to be replaced with an offset based equivalent before being copied.
This is done by overwriting the vector object with a \emph{CopyVector} object as shown in \autoref{lst:copyvector}.

\begin{lstlisting}[language=c++, numbers=left, caption=Component Class, captionpos=b, label={lst:copyvector}]
    struct CopyVector {
        size_t offset;
        size_t size;
    };
\end{lstlisting}

The \emph{CopyVector} struct consists of an \emph{offset} and a \emph{size} variable that are used to reference the underlying data of the original vector.
\emph{Offset} stores the offset of the vector's data relative to the start of the shared memory region, while \emph{size} stores the number of elements in the vector.

While recreating the component in the importing process, the \emph{CopyVector} can simply be replaced by a regular vector again, using the copied data by resolving the offset and size.

\subsection{Importing Components}
Since the component tree is transformed into a single contiguous memory block and all regular pointers are replaced by offset based pointers, simply using \lstinline{memcpy()} to import the topology into the processes private memory will not work.

Importing the component tree into the receiving processes memory is done as follows:

\begin{enumerate}
    \item Reading the \emph{Components} \emph{componentType} to determine its type.
    \item Recreating the \emph{Component} using the copy constructor of the correct component subclass.
    \item Recreating the \emph{attrib} map by inserting all copied key-value pairs.
    \item Recursively copying the children and recreating the \emph{children} vector.
\end{enumerate}

To recreate the \emph{children} vector, the offsets of the children stored in the \emph{CopyVector}, as described in \autoref{subsection:export_components},
need to replaced with pointers to their respective memory addresses in the importing process.

\section{Attribs}\label{section:attribs}
[Example and default attribs?] %TODO
Apart from the predefined properties each component subclass uses to represent the underlying hardware, Components and DataPaths have an \emph{attrib} map storing key-value pairs that can be used to add arbitrary data of any size.
It is implemented as a \lstinline{std::map<std::string, void*>}, mapping strings to void pointers.

This allows for a high degree of customization as there are no restrictions to the size or type of data. A string can just as easily be stored as a complex user-defined class.
However, the highly versatile use of the \emph{attrib} map also makes it difficult to copy the attrib map, since the size of the data is not necessarily known in advance.

\subsection{Packing and Exporting Attribs}\label{subsection:packing_and_exporting}
User-defined attributes can have any size and are not necessarily stored contiguously, which makes copying them into the shared memory region difficult.

A simple example for this problem is an \emph{attrib} storing a linked list. Each item in the list is stored in a different memory location,
the \emph{attrib}'s pointer pointing to the first element. The memory utilization of the list also directly depends on its size, which may change during the runtime of the program.

If an \emph{attrib} needs to be exported along with the component, the user needs to supply its size and provide the data as a single contiguous memory block.
In the example of the linked list, the list could be transformed into an array containing the same data.

For this purpose the user passes a pointer to function that transforms the \emph{attrib} into a \emph{CopyAttrib} struct as shown in \autoref{lst:copyattrib}.
\emph{CopyAttribs} consist of the attribs \emph{key}, a \emph{size} variable storing the datas total size in byte and a pointer to the linearized data block.

\begin{figure}[!ht] %TODO Fix position
    \includegraphics[scale=0.1, angle=90]{images/packing.jpg} %TODO Replace graphic
    \centering
    \caption{Packing a Linked List}
    \label{figure:packing_example}
\end{figure}

\autoref{figure:packing_example} illustrates using the \emph{pack} function supplied by the user
to transform an \emph{attrib} pointing to a linked list into a \emph{CopyAttrib} containing the \emph{attribs} key,
its data as a contiguous memory block and the datas size in byte.

\begin{lstlisting}[language=c++, numbers=left, caption=CopyAttrib Struct, captionpos=b, label={lst:copyattrib}]
    struct CopyAttrib {
        std::string key;
        size_t size;
        void* data;
    };
\end{lstlisting}

\autoref{lst:packing_function} shows how the user-defined \emph{pack} function for the linked list example could be implemented.
It takes a \lstinline|std::pair<std::string, void*>| containing an attribs key-value pair as a parameter and returns a \emph{CopyAttrib} containing the packed attrib.

In line 2, the \emph{pack} function compares the supplied key with all defined attrib keys to determine, what kind of data the attrib contains.
In this case, the key \lstinline|"EXAMPLE_KEY"| is associated with a linked list of \emph{ints}.
A new \emph{int} array of the same size as the list is allocated in line 4 to store the lists elements in a contiguous memory block.
In lines 7-9 the array is then filled by iterating over the list and placing it elements in the array one by one.
Finally, a \emph{CopyAttrib} consisting of the original key, the arrays size and a pointer to the array is returned in line 11.

Depending on the associated data, not every attrib is necessarily relevant to the receiving process.
If an attrib doesn't need to be exported with the rest of the component, the \emph{pack} function can simply return \lstinline|{key, 0, nullptr}| as shown in line 13
to indicate that the data belonging to a specific key ode not need to be copied.

\begin{lstlisting}[language=c++, numbers=left, caption=Example Packing Function, captionpos=b, label={lst:packing_function}]
  CopyAttrib pack(std::pair<std::string, void*> attrib) {
    if (!attrib.first.compare("EXAMPLE_KEY")) {
        std::list<int>* list = (std::list<int>*)attrib.second;
        int* arr = new int[list->size()];
        auto i = 0;

        for (const auto& item : *list) {
            arr[i] = item;
        }

        return {attrib.first, sizeof(int) * list->size(), arr};
    }
    return {key, 0, nullptr};
  }
\end{lstlisting}

To export the components attribs, they need to be arranged into a contiguous memory block.
This is done by writing the individual \emph{CopyVectors} provided by the \emph{pack} function sequentially into the shared memory segment.

\autoref{figure:attrib_memory} shows how the linearized attrib map is arranged in the shared memory.

As the number and size of the attribs varies heavily, the linearized attrib map is preceded by a \lstinline|size_t num_attribs| variable in the shared memory,
as illustrated in \autoref{figure:attrib_memory}.
The \lstinline|num_attribs| value indicates the number of attribs exported from the components map to the importing process.

\begin{figure}[!ht]
    \includegraphics[scale=0.1]{images/attrib_memory.jpg} %TODO Replace graphic
    \centering
    \caption{Attribs in Shared Memory}
    \label{figure:attrib_memory}
\end{figure}

\subsection{Unpacking and Importing Attribs}
To import the components attribs, the original attrib map needs to be recreated in the receiving process. This is done as follows:

\begin{enumerate}
    \item Reading the number of attribs from the top of the components attrib memory region as illustrated in \autoref{figure:attrib_memory}.
    \item Iterating over the attribs, reading the key string and the size of the data.
    \item Recreating the attrib key-value pair using the key and copying the data.
\end{enumerate}

After the linearized attrib map has been read and the attrib key-value pairs have been recreated in the importing process, the original state of the data has to be recreated.
For this purpose, the user has to supply an \emph{unpack} function with the signature shown in \autoref{lst:unpacking_function}.
The unpack function takes the \lstinline|std::pair<std::string, void*>| created earlier, which consists of the attrib key and a pointer to a single contiguous memory block
and returns the attrib in its original state, as it was before the \emph{pack} function was called by the exporting process.

\begin{lstlisting}[language=c++, numbers=none, caption=Unpack Function Declaration, captionpos=b, breaklines=true, label={lst:unpacking_function}]
    std::pair<std::string, void *> unpack(size_t size, std::pair<std::string, void *> attrib);
\end{lstlisting}

In the example of the linked list used in \autoref{subsection:packing_and_exporting}, the \emph{unpack} function would receive a key value pair containing the attrib key
and the \emph{int} array created by the \emph{pack} function earlier as well as the total size of the array in byte.

The \emph{unpack} function would then recreate the original linked list by reading the individual elements of the array and return the original attrib key value pair containing
the key string and the pointer to the linked list.

\section{DataPaths}
Apart from the component tree, sys-sage topologies, consist of \emph{DataPaths}.
DataPaths are used to connect two components of the component tree, to represent arbitrary relations between them.

\autoref{lst:datapath_class} shows part of the DataPath class implementation.
DataPaths have \emph{source} and \emph{target} component pointers, as shown in line 6 and 7, that represent the components connected by the DataPath.

Additionally, DataPaths can either be oriented, meaning that the source and target components are differentiated, or bidirectional, which is indicated by the \emph{oriented} variable.
The \emph{dp\_type} variable can be used to attach additional information about the nature of the relation between the components, such as a physical hardware connection or a shared cache.

Just like components, DataPaths have an \emph{attrib} map, which can be used to attach arbitrary data to the DataPaths using key-value pairs.

\begin{lstlisting}[language=c++, numbers=left, caption=DataPath Class, captionpos=b, label={lst:datapath_class}]
    class DataPath {
      public:
        map<string, void*> attrib;

      private:
        Component* source;
        Component* target;

        const int oriented;
        const int dp_type;
    };
\end{lstlisting}

\subsection{Exporting DataPaths}\label{subsection:export_dps}
Since DataPaths are used to connect two components, they only need to be exported if both the source and target component are exported. %TODO align
While exporting the component tree, each components DataPaths will be put in a \lstinline|std::map<DataPath*, std::pair<size_t, size_t>>|,
mapping the DataPath to a \lstinline|std::pair<size_t, size_t>|, storing the offsets of the DataPaths components in the shared memory region.

After the entire component tree is exported, it can easily be determined, which DataPaths need to be exported, by checking if both component offsets are set.

The actual export process for the DataPaths works as follows:

\begin{enumerate}
    \item Writing the total number of DataPaths exported to the shared memory region.
    \item Writing the offsets of the DataPaths source and target components, relative to the top of the memory mapped file, to the shared memory.
    \item Exporting the actual DataPath object using \lstinline{memcpy()}.
    \item Exporting the DataPaths \emph{attribs} using the provided \emph{pack} function as described in \autoref{section:attribs}.
\end{enumerate}

\begin{figure}[!ht] %TODO Fix position
    \includegraphics[scale=0.15]{images/export_dps.jpg} %TODO Replace graphic
    \centering
    \caption{DataPaths in Shared Memory}
    \label{figure:export_dps}
\end{figure}

\autoref{figure:export_dps} illustrates how the DataPaths are arranged in the shared memory file.

\subsection{Importing DataPaths}
To import the DataPaths into the local memory of the receiving process, the steps described in \autoref{subsection:export_dps} need to be reversed:

\begin{enumerate}
    \item Reading the total number of DataPaths from the top of the DataPaths memory region.
    \item Iterating over the shared memory block reading the source and target offsets and recreating the DataPath object using its copy constructor.
\end{enumerate}

Since the components have been imported into the local memory and thus have a new memory location, the read offsets need to be translated to regular pointers
pointing to the components new memory location.

For this purpose, a \lstinline|std::map<size_t, Component*>| is created while exporting the component tree, mapping the components offset in the shared memory region to its
memory location in the importing process.

\begin{enumerate}
    \setcounter{enumi}{2}
    \item Recreating the DataPaths source and target pointers and adding the DataPath to the components DataPath vectors.
    \item Importing the DataPaths \emph{attribs} using the provided \emph{unpack} function as described in \autoref{section:attribs}.
\end{enumerate}