\chapter{Conclusion and Future Work}
In this thesis, a file backed shared memory system was implemented,
which expands the existing sys-sage library to allow users to share sys-sage topologies between processes and nodes in HPC systems.

The systems creates memory mapped files, shared between different processes, which can then be used to transfer topologies between the processes.
To do this, all necessary components, DataPaths and attribs need to be copied into the shared memory region by the exporting process.
Any process wishing to import the topology can then use this memory region to recreate the topology in its local memory.

One of the main drawbacks of the shared memory system implemented in this thesis is its inability to update shared topologies once they have been exported.
This could be an issue for sys-sage users, as topologies are often prone to frequent changes and updates, such as regular measurements being saved in a components attribs.
As it stands, each process would either have to provide its own version of the necessary measurements, or the the components in question would have to be exported after every update.
Either solution could have significant impact on the systems performance, depending on the size of the components and the frequency of the updates.

In the future, it could be beneficial to implement additional functionality allowing users to update existing shared topologies without having to redo the entire export.
Ideally, every process would be able to create updates to shared topologies, not just the process that originally exported the topology.

This would make it much easier for users to keep all their processes up to date on the topology of the HPC system.

Additionally, it would be convenient if shared component subtrees could be automatically integrated into the topology of receiving processes.
This would make it much easier for users to share smaller parts of topologies, without having to add it to the topology manually.

Despite these drawbacks, the shared memory implementation works as intended and provides the necessary tools to share topologies within HPC systems.